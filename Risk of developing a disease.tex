\documentclass[12pt]{article}
\usepackage[utf8]{inputenc}
\usepackage[T1]{fontenc}
\usepackage{graphicx}
\usepackage[export]{adjustbox}
\graphicspath{ {./} }
\usepackage{amsmath}
\usepackage{amsfonts}
\usepackage{amssymb}
\usepackage[version=4]{mhchem}
\usepackage{stmaryrd}
\usepackage{titlesec}
\usepackage{geometry}
\usepackage{relsize}
\usepackage{indentfirst} 
\usepackage{hyperref}
\usepackage{natbib}



\begin{document}
\title{Capstone Project Proposal: Risk of developing a disease}
% \maketitle

\section{Introduction}

\section{Problem Statement}



\section{Datasets and Inputs}

For the computation of non-negative matrix factorization, we utilize the matrix $ A$.


\section{Solution Statement}

Our objective is to perform non-negative matrix factorization on the matrix $A$ itself using Hamiltonian simulation and phase estimation to determine disease progression risks. implementable in quantum circuits.

% \section{Risk of developing a disease}

Our planned primary method is QDCA \cite{du2018quantum}, Quantum divide-and-conquer anchoring, which is derived from the classical DCA algorithm and greatly reduces the runtime required to extract partial representations based on large-scale matrices. Similar to DCA, QDCA can exponentially accelerate the approximate separable NMF problem. It decomposes DCA into s subproblems corresponding to s random projections, consisting of divide and conquer parts. The divide part involves breaking down DCA into s subproblems associated with s random projections, where the i-th subproblem is composed of the i-th random projection and the subsequent process to determine $\bar{A}^{i}$. The elements in each subproblem's $X \beta_{i}$ are encoded as probability amplitudes of $\left|\psi_{i}\right\rangle$. Then, a heuristic post-selection method called "after-selection" is employed, which consumes N copies of $|\psi_{i}\rangle$ to achieve a logarithmic runtime complexity and locate an anchor index in the projection space corresponding to the maximum probability amplitude of $|\psi_{i}\rangle$. In the conquer part, we collect $\left\{\bar{A}^{i}\right\}_{i=1}^{s}$ in classical form and use the sorting algorithm \cite{knuth1997art} to sort the s indices in $\left\{\bar{A}^{i}\right\}_{i=1}^{s}$ in $O(s \log s)$ time. We then select the top r indices as R. With the selected R, we obtain the decomposed matrix $X(R,:)$, which allows us to factorize $X$ as $X=F X(R,:)$, where the basis matrix $X(R,:)$ consists of the r rows of X, and F is the non-negative encoding matrix. These steps primarily consist of four stages.

\subsection{Step 1: Quantum Simulation}

First, we encode the given classical data matrix A and random vectors $\beta_i$ into quantum states.

1.1 We get the data matrix $\tilde{X} \in \mathbb{R}^{n \times m}$

1.2 Embed $\tilde{X}$ into a high-dimensional matrix using an extension matrix, i.e.,

$$
X:=\left(\begin{array}{cc}
0 & \tilde{X} \\
\tilde{X}^{\dagger} & 0
\end{array}\right) \in \mathbb{C}^{(n+m) \times(n+m)}
$$

It is a square matrix and conjugate transpose symmetric.

1.3 Construct a unitary time evolution $e^{i X t}$ for simulating time t based on the method proposed by Berry \cite{berry2007efficient} and Harrow \cite{harrow2009quantum}

1.3.1 Modified permutation matrix $S_{X} \in \mathbb{C}^{(n+m)^{2} \times(n+m)^{2}}$:

Efficiently embed X into a large sparse matrix $S_{X} \in \mathbb{C}^{(n+m)^{2} \times(n+m)^{2}}$.

$$
S_{X}:=\sum_{i, j=1}^{n+m} X_{i, j}|i\rangle\langle j|\otimes| j\rangle\langle i| \in \mathbb{C}^{(n+m)^{2} \times(n+m)^{2}}
$$

where $X_{i, j}$ is the (i, j)-th element of X.

1.3.2 Unitary time evolution $e^{i X t}$:

For small time intervals $\Delta t$, apply $e^{i S_{X} \Delta t}$ on a larger system $\rho \otimes \sigma$ to simulate the evolution $e^{i X t /(n+m)}$ on the $\sigma$ system, with negligible error. The factor $1 /(n+m)$ is a scaling factor that does not affect the final result. Therefore, what we need to do is to apply $e^{i S_{X} \Delta t}$ on $\rho \otimes \sigma$, where the auxiliary state $\rho=|\overrightarrow{1}\rangle\langle\overrightarrow{1}|$ and $|\overrightarrow{1}\rangle=\frac{1}{\sqrt{n+m}} \sum_{i}|i\rangle$. This achieves the simulation of the unitary time evolution $e^{i X t}$ on a quantum circuit, with a time complexity of $O($poly $\log (n+m))$.

$$
\begin{aligned}
& \operatorname{tr}_{\rho} e^{i S_{X} \Delta t} \rho \otimes \sigma e^{-i S_{X} \Delta t} \\
& =e^{i \frac{X}{(n+m)} \Delta t} \sigma e^{-i \frac{X}{(n+m)} \Delta t}+O\left(\Delta t^{2}\right)
\end{aligned}
$$

Therefore, what needs to be constructed is a quantum circuit that simulates $e^{i S_{X} t}$.

1.3.3 Preparation of random vectors $\beta$:

Prepare a normalized vector $\beta$ as a quantum state $|\beta\rangle$ based on the methods proposed by Grover and Rudolph \cite{grover2002creating} and Harrow\cite{lloyd2014quantum}:

$\beta=\left[\alpha_{0}, \cdots, \alpha_{(n+m)-1}\right]^{\top} \in \mathbb{C}^{(n+m)}$

$|\beta\rangle=\sum_{i=1}^{n+m} \alpha_{i}|i\rangle$.

Therefore, the quantum state $|\beta_{i}\rangle$ encoding vector $\beta_{i}$ can be obtained.

\subsection{Step 2: Quantum Linear Operations}

Using the Quantum Principal Component Analysis (QPCA) scheme proposed by Lloyd \cite{lloyd2014quantum} and the subsequent phase estimation algorithm by Shor \cite{shor1999polynomial}, obtain a quantum state $|\psi_{i}\rangle$ that is proportional to the random projection $X\beta_{i}$.

2.1 Perform the eigenvalue decomposition of $X$ as $\sum_{j} \lambda_{j}|u_{j}\rangle\langle u_{j}|$, where $\lambda_{j}$ and $|u_{j}\rangle$ represent the eigenvalues and corresponding eigenvectors, respectively.

2.2 Apply the exponential matrix $e^{i X t_{0} /(n+m)}$ to $|\beta_{i}\rangle$ with a time complexity of $O($poly $\log (n+m))$,

obtaining

$$
\Lambda_{q}\left(e^{i X t_{0} /(n+m)}\right)|k\rangle|\beta_{i}\rangle=\frac{1}{\sqrt{2^{q}}} \sum_{k}|k\rangle e^{i k X t_{0} /(n+m)}|\beta_{i}\rangle
$$

where $\Lambda_{q}(\cdot)$ applies $e^{i X t_{0}}$ to $|\beta_{i}\rangle$ q times.

2.3 Eigenvalue register $|k\rangle$, with a time complexity of $O(1 / \epsilon)$.

Using $|k\rangle$ as the eigenvalue register, we can obtain the quantum state

$$
\frac{1}{\sqrt{\sum_{j}|\beta_{j}|^{2}}} \sum_{\frac{|\lambda_{j}|}{(n+m)} \geq \epsilon} \beta_{j}\left|\frac{\lambda_{j}}{(n+m)}\right\rangle|u_{j}\rangle
$$

where $\beta_{j}=\langle u_{j}|\beta_{i}\rangle$, with a time complexity of $O(1 / \epsilon)$.

To obtain a quantum form similar to $X\left|\beta_{i}\right\rangle=\sum_{j} \lambda_{j}\left\langle u_{j}|\beta_{i}\right\rangle|u_{j}\rangle=\sum_{j} \lambda_{j}\beta_j|u_{j}\rangle$, corresponding to the result of the random projection $X\beta_{i}$, the eigenvalues are extracted as probability amplitudes of the quantum state.

2.4 Controlled-rotation and non-computation

Introduce an auxiliary qubit to perform controlled-rotation on $|\lambda_{j}\rangle$ and non-computation on the eigenvalue register. The resulting quantum state is proportional to

$$
\sum_{\frac{\left|\lambda_{j}\right|}{(n+m)} \geq \epsilon} \beta_{j}\left|u_{j}\right\rangle \left(\sqrt{1-\frac{\lambda_{j}^{2}}{C^{2}(n+m)^{2}}}|0\rangle +\frac{\lambda_{j}}{C(n+m)}|1\rangle\right)
$$

where $C=O\left(1 / \lambda_{\max }\right)$, and $\lambda_{\max }$ is the maximum eigenvalue of $X$.

2.5 Measurement of the last qubit

Measure the last qubit and observe that under the condition of observing 1, the resulting quantum state is proportional to $X\beta_{i}$, i.e.,

$$
|\psi_{i}\rangle=\frac{1}{\sqrt{\sum_{j} \frac{|\beta_{j} \lambda_{j}|^{2}}{C^{2}(n+m)^{2}}}} \sum_{j} \beta_{j} \frac{\lambda_{j}}{C(n+m)}|u_{j}\rangle .
$$

2.6 Computational Complexity

The complexity of computing $X\beta_{i}$ for $i=\{1, \ldots, s\}$ is $O($poly $\log (n+m) / \epsilon)$, assuming an expected error of $1 / \epsilon=O($poly $\log (n+m))$. Therefore, the overall complexity of computing $X\beta_{i}$ for $i=\{1, \ldots, s\}$ is $O($poly $\log (n+m))$.

\subsection{Step 3: Heuristic Post-Selection}

The elements of $X\beta_{i}$ are encoded as probability amplitudes of $|\psi_{i}\rangle$. The heuristic post-selection is employed to retain the quantum advantage of reading all probability amplitudes by consuming N copies of $|\psi_{i}\rangle$, achieving a logarithmic runtime complexity. This process allows us to identify an anchor index in the projection space (corresponding to the maximum probability amplitude of $|\psi_{i}\rangle$) by performing measurements. The most frequently occurring index in the N outputs is highly likely to be the anchor index in the projection space. The paper suggests that consuming N=O(poly $\log (n+m))$ copies of $|\psi_{i}\rangle$ is sufficient to determine the anchor index in the projection space $\bar{A}^{i}$ with a high probability.

Given N copies of $|\psi_{i}\rangle$, the process of heuristic after-selection is as follows:

1. Measure each $|\psi_{i}\rangle$ in the computational basis.

2. Record the most frequent index among the N outputs as the anchor index in the projection space.

This completes the divide step of QDCA. Since QDCA only obtains the index with the maximum absolute value, the number of random vectors $\beta_{i}$ should be doubled. Since $s \ll \min (n, m)$, we have $2 s \ll n+m$, and $s$ is still $O(r \log r)$. Therefore, including the runtime for searching the most frequent index, the complexity of this step is $O($poly $\log (n+m))$.

\subsection{Step 4: Classical Conquer Step}

We collect $\left\{\bar{A}^{i}\right\}_{i=1}^{s}$ in classical form. Using the sorting algorithm Knuth \cite{knuth1997art}, we can sort the s indices in $\left\{\bar{A}^{i}\right\}_{i=1}^{s}$ in $O(s \log s)$ time. After sorting, we select the top r indices as R, which are the most frequently occurring indices among all s subproblems. With the selected R, we obtain the decomposed matrix $X(R,:)$, allowing us to factorize $X$ as $X=F X(R,:)$, where the basis matrix $X(R,:)$ consists of the r rows of X, and F is the non-negative encoding matrix. Since $s=O(r \log r)$, the runtime complexity of this step is $O($poly $\log (n+m))$.



\section{A Simple Test Case}

Assuming that the matrix A we need to decompose is a 4x4 matrix:

$$
A = \begin{bmatrix}
0 & 0 & 0 & 0 \\
0 & 1 & 0 & 0 \\
0 & 0 & 2 & 0 \\
0 & 0 & 0 & 2 \\
\end{bmatrix}
$$

Its eigenvalues are:

$$
\lambda_0 = 0, \lambda_1 = 1, \lambda_2 = 2, \lambda_3 = 2
$$

Corresponding eigenvectors are:

$$
v_0 = \begin{bmatrix}
1 \\
0 \\
0 \\
0
\end{bmatrix},
v_1 = \begin{bmatrix}
0 \\
1 \\
0 \\
0
\end{bmatrix},
v_2 = \begin{bmatrix}
0 \\
0 \\
1 \\
0
\end{bmatrix},
v_3 = \begin{bmatrix}
0 \\
0 \\
0 \\
1
\end{bmatrix}
$$

Let's assume our random vector $\beta_i$ is:

$$
\beta_{i} =
\begin{bmatrix}
\alpha_0 \\
\alpha_1 \\
\alpha_2 \\
\alpha_3
\end{bmatrix}
=
\begin{bmatrix}
\frac{1}{2} \\
\frac{1}{2} \\
\frac{1}{2} \\
\frac{1}{2}
\end{bmatrix}
$$

We can prepare it as a quantum state:

$$
|\psi_0\rangle=|\beta_{i}\rangle=\alpha_0 |00\rangle +\alpha_1 |01\rangle +\alpha_2 |10\rangle +\alpha_3 |11\rangle\\
=\frac{1}{2} |00\rangle +\frac{1}{2} |01\rangle +\frac{1}{2}  |10\rangle +\frac{1}{2}  |11\rangle
$$

The corresponding quantum circuit is as follows:

% \vspace{\baselineskip}
\begin{figure}[ht]
\begin{center}
\includegraphics[max width=0.4\textwidth]{./01.png} 
% \caption{A polished silicon wafer} 
\end{center}
\end{figure}
% \vspace{\baselineskip}

First, using the oracle $\Lambda_{q}(\cdot)$, for simulation time $t_{0}$, we apply the exponential matrix $e^{i A t_{0} /(n+m)}$ to $|\beta_{i}\rangle$, resulting in

$$
\Lambda_{q}\left(e^{i A t_{0} /(n+m)}\right)|k\rangle|\beta_{i}\rangle=\frac{1}{\sqrt{2^{q}}} \sum_{k}|k\rangle e^{i k A t_{0} /(n+m)}|\beta_{i}\rangle
$$

where $q$ is a positive integer, $|k\rangle$ consists of $q$ quantum bits (i.e., $|k\rangle=|0\rangle^{\otimes q}$), and the oracle $\Lambda_{q}(\cdot)$ applies $e^{i A t_{0}}$ to $|\beta_{i}\rangle$ for $k$ times. This equation can be seen as a form of quantum Fourier transform.

By adding a feature value register with q quantum bits and applying the exponential matrix $e^{i A t_{0} /(n+m)}$ constructed unitary gates, we have:




$$
|\psi_1\rangle=\frac{1}{2} \left[|00\rangle e^{i 0 A t_{0} /4} + |01\rangle e^{i 1 A t_{0} /4} + |10\rangle e^{i 2 A t_{0} /4} + |11\rangle e^{i 3 A t_{0} /4}\right] |\beta_{i}\rangle
$$

$$
=\frac{1}{2} \left[|00\rangle + e^{i A t_{0} /4} |01\rangle + e^{i 2 A t_{0} /4} |10\rangle + e^{i 3 A t_{0} /4} |11\rangle\right] |\beta_{i}\rangle
$$

The quantum circuit is as follows:

% \vspace{\baselineskip}
\begin{figure}[ht]
\begin{center}
\includegraphics[max width=0.8\textwidth]{./02.png} 
% \caption{A polished silicon wafer} 
\end{center}
\end{figure}
% \vspace{\baselineskip}

Next, by taking $|k\rangle$ as the eigenvalue register, we can obtain the quantum state through quantum operations:

$$
\frac{1}{\sqrt{\sum_{j}|\beta_{j}|^{2}}} \sum_{\frac{|\lambda_{j}|}{(n+m)} \geq \epsilon} \beta_{j}\left|\frac{\lambda_{j}}{(n+m)}\right\rangle|u_{j}\rangle
$$

where $\beta_{j}=\langle u_{j}|\beta_{i}\rangle$, and the required time is $O(1 / \epsilon)$. In order to obtain a quantum form similar to $X\left|\beta_{i}\right\rangle=\sum_{j} \lambda_{j}\left\langle u_{j}|\beta_{i}\right\rangle|u_{j}\rangle$, corresponding to the result of the random projection $X\beta_{i}$, the eigenvalues will be extracted as the probability amplitudes of the quantum state.

$\beta_j = \langle u_{j}|\beta_i\rangle$ can be computed as:

$$
\beta_0 = \langle u_{0}|\beta_i\rangle = \begin{bmatrix}
1 & 0 & 0 & 0
\end{bmatrix}
\begin{bmatrix}
\alpha_0 \\
\alpha_1 \\
\alpha_2 \\
\alpha_3
\end{bmatrix} = \alpha_0=\frac{1}{2}
$$

$$
\beta_1 = \langle u_{1}|\beta_i\rangle = \begin{bmatrix}
0 & 1 & 0 & 0
\end{bmatrix}
\begin{bmatrix}
\alpha_0 \\
\alpha_1 \\
\alpha_2 \\
\alpha_3
\end{bmatrix} = \alpha_1=\frac{1}{2}
$$

$$
\beta_2 = \langle u_{2}|\beta_i\rangle = \begin{bmatrix}
0 & 0 & 1 & 0
\end{bmatrix}
\begin{bmatrix}
\alpha_0 \\
\alpha_1 \\
\alpha_2 \\
\alpha_3
\end{bmatrix} = \alpha_2=\frac{1}{2}
$$

$$
\beta_3 = \langle u_{3}|\beta_i\rangle = \begin{bmatrix}
0 & 0 & 0 & 1
\end{bmatrix}
\begin{bmatrix}
\alpha_0 \\
\alpha_1 \\
\alpha_2 \\
\alpha_3
\end{bmatrix} = \alpha_3=\frac{1}{2}
$$

Let's assume $\epsilon=0.1$, then we can obtain:

$$
|\psi_{2}\rangle
= \frac{1}{\sqrt{\alpha_1^2 + \alpha_2^2 + \alpha_3^2}} \left(\alpha_1\left|\frac{1}{4}\right\rangle e^{i A t_{0} 1/4} |u_{1}\rangle + \alpha_2\left|\frac{2}{4}\right\rangle e^{i A t_{0} 2/4} |u_{2}\rangle +  \alpha_3\left|\frac{2}{4}\right\rangle e^{i A t_{0} 3/4} |u_{3}\rangle\right)
$$
$$
= \frac{2}{\sqrt{3}}\left(\frac{1}{2}\left|\frac{1}{4}\right\rangle e^{i A t_{0} 1/4} |u_{1}\rangle + \frac{1}{2}\left|\frac{2}{4}\right\rangle e^{i A t_{0} 2/4} |u_{2}\rangle +  \frac{1}{2}\left|\frac{2}{4}\right\rangle e^{i A t_{0} 3/4} |u_{3}\rangle\right)
$$

% \vspace{\baselineskip}
\begin{figure}[ht]
\begin{center}
\includegraphics[max width=0.8\textwidth]{./03.png} 
% \caption{A polished silicon wafer} 
\end{center}
\end{figure}
% \vspace{\baselineskip}

Then we introduce an ancillary qubit, rotate $|\lambda_{j}\rangle$ conditionally, and uncompute the eigenvalue register. The resulting quantum state is proportional to:

$$
\begin{aligned}
\sum_{\frac{|\lambda_{j}|}{(n+m)} \geq \epsilon}
\beta_{j}|u_{j}\rangle & \left(\sqrt{1-\frac{\lambda_{j}^{2}}{C^{2}(n+m)^{2}}}|0\rangle\right.  \left.+\frac{\lambda_{j}}{C(n+m)}|1\rangle\right)
\end{aligned}
$$

where $C=O\left(1 / \lambda_{\max }\right)$, and $\lambda_{\max }$ is the maximum eigenvalue of $X$.

$\lambda_{\max }$=2, therefore we set $C=1/2$. By applying the rotation and uncomputing operations, we obtain:


$$
\begin{aligned}
|\psi_{3}\rangle
&= \frac{2}{\sqrt{3}}\left(\frac{1}{2}|u_{1}\rangle\left(\sqrt{1-\frac{1^2}{(1/2)^2(4)^2}}|0\rangle + \frac{1}{(1/2)(4)}|1\rangle\right) \\
&+ \frac{1}{2}|u_{2}\rangle\left(\sqrt{1-\frac{2^2}{(1/2)^2(4)^2}}|0\rangle \right. \quad + \left. \frac{2}{(1/2)(4)}|1\rangle\right) \\
&+ \frac{1}{2}|u_{3}\rangle\left(\sqrt{1-\frac{2^2}{(1/2)^2(4)^2}}|0\rangle \right. \quad + \left. \frac{2}{(1/2)(4)}|1\rangle\right)\right))
\end{aligned}
$$



$$
=\frac{2}{\sqrt{3}}\left(\frac{1}{2}|u_{1}\rangle\left(\frac{\sqrt{3}}{2}|0\rangle + \frac{1}{2}|1\rangle\right) + \frac{1}{2}|u_{2}\rangle|1\rangle+ \frac{1}{2}|u_{3}\rangle|1\rangle\right)
$$

% \vspace{\baselineskip}
\begin{figure}[ht]
\begin{center}
\includegraphics[max width=0.8\textwidth]{./04.png} 
% \caption{A polished silicon wafer} 
\end{center}
\end{figure}
% \vspace{\baselineskip}

Upon measuring the last qubit, the ancillary qubit, and observing the outcome 1, we can see that the final output quantum state is proportional to $X\beta_{i}$, which is given by:

$$
|\psi_{i}\rangle=\frac{1}{\sqrt{\sum_{j} \frac{|\beta_{j} \lambda_{j}|^{2}}{C^{2}(n+m)^{2}}}} \sum_{j} \beta_{j} \frac{\lambda_{j}}{C(n+m)}|u_{j}\rangle .
$$

Since:

$$
\sqrt{\sum_{j} \frac{|\beta_{j} \lambda_{j}|^{2}}{C^{2}(n+m)^{2}}} = \sqrt{\frac{|\frac{1}{2} \cdot 1|^{2}}{4^2} + \frac{|\frac{1}{2} \cdot 2|^{2}}{4^2} + \frac{|\frac{1}{2} \cdot 2|^{2}}{4^2}} = \sqrt{\frac{1}{16} + \frac{4}{16} + \frac{4}{16}} = \sqrt{\frac{9}{16}} = \frac{3}{4}
$$

we have:

$$
|\psi_{4}\rangle=\frac{4}{3}\left(\frac{1}{2} \cdot \frac{1}{4}\cdot \frac{1}{2}|u_{1}\rangle + \frac{1}{2} \cdot \frac{2}{4}\cdot \frac{1}{2}|u_{2}\rangle + \frac{1}{2} \cdot \frac{2}{4}\cdot \frac{1}{2}|u_{3}\rangle\right) \\

= \frac{1}{12}|u_{1}\rangle + \frac{1}{6}|u_{2}\rangle + \frac{1}{6}|u_{3}\rangle
$$

The quantum circuit is shown as follows:

% \vspace{\baselineskip}
\begin{figure}[ht]
\begin{center}
\includegraphics[max width=0.8\textwidth]{./05.png} 
% \caption{A polished silicon wafer} 
\end{center}
\end{figure}
% \vspace{\baselineskip}
    
This is the quantum state obtained when measuring the ancillary qubit and observing the outcome 1. It corresponds to:

$$
\frac{1}{12}v_1 + \frac{1}{6}v_2 + \frac{1}{6}v_3
$$

$$
= \frac{1}{12}\begin{bmatrix}
0 \\
1 \\
0 \\
0
\end{bmatrix} + \frac{1}{6}\begin{bmatrix}
0 \\
0 \\
1 \\
0
\end{bmatrix} + \frac{1}{6}\begin{bmatrix}
0 \\
0 \\
0 \\
1
\end{bmatrix}
$$

$$
= \begin{bmatrix}
0 \\
\frac{1}{12} \\
\frac{1}{6} \\
\frac{1}{6}
\end{bmatrix}
$$

We can see that it is proportional to:

$$
X\beta_{i}=\begin{bmatrix}
0 \\
\frac{1}{2} \\
1 \\
1
\end{bmatrix}
$$

By consuming multiple copies of such $|\psi_{i}\rangle$, we can record the most frequently occurring index among these outputs as the index corresponding to the largest component.

Using a heuristic post-selection approach, we collect $\left\{\bar{A}^{i}\right\}_{i=1}^{s}$ in classical form, and then select the top $r$ indices as $R$, which are the indices with the highest frequency among all $s$ subproblems. Thus, we obtain the decomposed matrix $X(R,:)$ as the basis matrix for the non-negative matrix decomposition of $X$, completing the decomposition process.







\section{Expected Outputs}

We will ultimately obtain a set of indices for the anchor points, and by using this index set, we can construct the basis matrix for the non-negative matrix decomposition of X, completing the decomposition process.

\section{Time Scheduling}
We plan to complete this project schedule within eight weeks and conduct a project demonstration in the ninth week. The following is the proposed timeline:

\textbf{Week 1: Project Kickoff and Literature Review}
\begin{itemize}
    \item Define project objectives and scope.
    \item Conduct a literature review and research on quantum computing, non-negative matrix factorization, and related concepts.
\end{itemize}

\textbf{Week 2: Data Collection and Processing}
\begin{itemize}
    \item Collect and process the dataset to be used.
    \item Determine the matrix A for computation.
    \item Configure the required quantum computing environment.
\end{itemize}

\textbf{Week 3: Step 1: Quantum Simulation - Part 1}
\begin{itemize}
    \item Implement Section 1.1 and 1.2, quantum encode the given classical data matrix A.
    \item Design and test a program to embed A into a higher-dimensional matrix.
\end{itemize}

\textbf{Week 4: Step 1: Quantum Simulation - Part 2}
\begin{itemize}
    \item Implement Section 1.3, construct a quantum circuit for simulating time evolution.
    \item Test and validate the accuracy of the quantum simulation.
\end{itemize}

\textbf{Week 5: Step 2: Quantum Linear Operations - Part 1}
\begin{itemize}
    \item Implement Section 2.1 and 2.2, use Quantum Principal Component Analysis (QPCA) to acquire quantum states.
    \item Begin work on eigenvalue decomposition and random projections.
\end{itemize}

\textbf{Week 6: Step 2: Quantum Linear Operations - Part 2}
\begin{itemize}
    \item Complete the implementation of Sections 2.3, 2.4, 2.5.
    \item Validate the results of quantum linear operations.
\end{itemize}

\textbf{Week 7: Step 3: Heuristic Post-Selection}
\begin{itemize}
    \item Implement and test the heuristic post-selection algorithm.
    \item Analyze and optimize the post-selection process.
\end{itemize}

\textbf{Week 8: Step 4: Classical Conquer Step and Result Analysis, Report Writing}
\begin{itemize}
    \item Implement the classical conquer step.
    \item Validate and adjust the results.
    \item Analyze and interpret the project outcomes.
    \item Write the project report.
\end{itemize}

\textbf{Week 9: Project Presentation}
\begin{itemize}
    \item Prepare and conduct the project presentation.
    \item Evaluate the success of the project and summarize the lessons learned and knowledge gained.
\end{itemize}


\bibliographystyle{IEEEtran}  
\bibliography{references} 


\end{document}